\documentclass{article}
\title{Algorithmic Music Composition}
\author{Ch.\ B. ten Brinke}

\date{\today}
\begin{document}

\maketitle

\begin{abstract}
Short introduction to subject of the paper \ldots 
\end{abstract}

\section{Introduction}
Algorithmic music composition is useful in several ways.
Not only may it provide convenience in producing musical compositions, it also may enhance our understanding of music through the models and methods we use.
Besides, think of the potential that realtime music generation has in video games.
Nowadays it is often hard to make the music react on what happens in the game.
It is mostly accomplished by making a switch to another (part of a) soundtrack as smoothly as possible or, the other way around, by making the game react on the soundtrack.
It would be helpful in this area if we really could generate decent music that corresponds to what happens in the game, on a realtime basis.

\section{Divide and Conquer (or Reductionism)}
A technique that is often used in mathematics to describe a complex phenomenon is Divide and Conquer.
In this technique the system is cut into smaller components, which are then 
investigated seperately.
Afterwards, the knowledge gained is put together again to make up a model for the system that were to be described.

This technique may be beneficial here.
To the author it seems that some aspects of musical harmony do allow to be studied independently.
Let us look at an example.
Consider the major dominant septime $E_7$ chord.
It solves well to the minor $A_m$ chord.
Now if we remove the tone $e$ in $E_7$ we get a $G_{\#dim}$ chord, which also solves to $A_m$, but less well.
The observation here is that removing a single tone from a chord doesn't necassarily change the musical meaning of the chord completely.

A second example.
Consider the chords $E_7$ and $F_{m6}$.
They both solve to $A_m$
Both consisting of four tones, they have only two tones in common, namely $g_\#$ and $d$.
This suggests that exactly these two tones are responsible for $E_7$ and $F_{m6}$ being able to solve to $A_m$.

If this were true in general, we could certainly benefit from studying the properties of small (parts of) musical objects.
Namely, the knowledge of the smaller components of a musical object would tell us how the bigger whole works.

\end{document}
