%        File: ContentsAndApproach.tex
%     Created: Mon May 28 10:00 PM 2012 C
% Last Change: Mon May 28 10:00 PM 2012 C
%
\documentclass[a4paper]{article}
\title{Contents and Approach}
\author{Chiel ten Brinke}
\begin{document}
\maketitle

\paragraph{Goals}
In this thesis we will study dissonance of intervals and chords in a mathematical way.

The main goal of mathematical studies upon this subject, is always formalizing harmony theory.
Nowadays, harmony theory is largely based on experience, instead of a mathematical system.
In spite of the fact that there is much order in music, it appears to be hard to invent a satisfying mathematical theory.
Many attempts have been made, with many different approaches.
In this thesis we like to contribute to this quest by studying the following two questions.
\begin{enumerate}
    \item How can we formalize numbertheoretical music theory about dissonance in modern mathematics?
    \item How can pure scales be aproximated by equal tempered scales?
\end{enumerate}

\paragraph{Storyline and approach} 
%%Dit is facultatief. Als we nog pagina's nodig hebben, kan dit wellicht van pas komen
%First we will give an overview of the different approaches on this subject.
%Next we will shortly cover the most important theories and state their differences.
%These studies allow us to have a good overview of music and harmony and the different ways to look at it.
%We might therefore be able to judge the different theories.

We will consider the tonal space as a group.
A group can be endowed with a metric, which we will interpret as a measure of dissonance.
Many metrics are possible, but we will mainly consider a variant on the formula that Leonhard Euler presented in his work \emph{Tentamen novae theoriae musicae ex certissismis harmoniae principiis dilucide expositae}.
%, Euler gives only little arguments for his assumptions concerning perception of harmony.
%We can attempt to make the theory a bit more intuitive by explaining assumptions more in detail.
%For now we will restrict ourselves here mostly to the theory about consonance of intervals, and in addition extend to chords consisting of more than two tones.
%We will then be able to review this theory and cover various critical comments on it.
%Finally a conclusion is presented, in which Euler's ideas are judged, based upon the preceding studies.
%We also might put here a few possible improvements and thoughts on Euler's theory.

After that, we don't want to stick to pure scales only.
We will show how pure scales can be aproximated by equal-tempered scales.
This will allow us to understand why some equal-tempered scales are more prominent than others, and give us a basic understanding how equal-tempered scales relate to pure scales.

\paragraph{Other remarks}
We want the thesis to be readable for any mathematician.
Therefore the musical concepts are defined rigorously, such that no musical pre-knowledge is required.

%In order to judge music theory, the theory has to be tested against real music.
%However, we do not claim any truth of the metrics we use.
%Any metric could be used.
%The theory is not tested on complete musical compositions, as we won't pay much attention to sequential harmony.
%Instead the theory is tested on single samples such as intervals and chords.

\end{document}
